\documentclass[DM,lsstdraft,STR,toc]{lsstdoc}
\usepackage{geometry}
\usepackage{longtable,booktabs}
\usepackage{enumitem}
\usepackage{arydshln}

\input meta.tex

\providecommand{\tightlist}{
  \setlength{\itemsep}{0pt}\setlength{\parskip}{0pt}}

\setcounter{tocdepth}{4}

\begin{document}

\def\milestoneName{Dm-Suit-8 Suit Portal Integrated With Workspace}
\def\milestoneId{LVV-P69}
\def\product{LSP Services}

\setDocCompact{true}

\title{ LVV-P69 Dm-Suit-8 Suit Portal Integrated With Workspace Test Plan and Report}
\setDocRef{\lsstDocType-\lsstDocNum}
\date{\vcsdate}
\setDocUpstreamLocation{\url{https://github.com/lsst/lsst-texmf/examples}}
\author{ Gregory Dubois-Felsmann }

\input history_and_info.tex


\setDocAbstract{
This is the test plan and report for LVV-P69 (Dm-Suit-8 Suit Portal Integrated With Workspace),
an LSST level 2 milestone pertaining to the Data Management Subsystem.
}


\maketitle

\section{Introduction}
\label{sect:intro}


\subsection{Objectives}
\label{sect:objectives}

 Demonstrate that the Portal Aspect software has the ability to read from
and write to a WebDAV server in the LSP.\\
This test plan performs only a partial verification of the applicable
requirements, as this Level 3 milestone was understood to represent
demonstration of the ability to interface with a WebDAV service, not the
LSP-system-level test of the full role of the User File Workspace in the
LSP design.\\[2\baselineskip]Note that at the time of testing, only a
demonstration-level WebDAV service was available, not integrated with
the Notebook Aspect user file system.



\subsection{System Overview}
\label{sect:systemoverview}

 \textbf{Applicable Documents:}\\[2\baselineskip]\citeds{LDM-554} Data Management
LSST Science Platform Requirements


\subsection{Document Overview}
\label{sect:docoverview}

This document was generated from Jira, obtaining the relevant information from the 
\href{https://jira.lsstcorp.org/secure/Tests.jspa#/testPlan/LVV-P69}{LVV-P69}
~Jira Test Plan and related Test Cycles (
  \href{https://jira.lsstcorp.org/secure/Tests.jspa#/testCycle/LVV-C151}{LVV-C151}
).

Section \ref{sect:intro} provides an overview of the test campaign, the system under test (\product{}),
the applicable documentation, and explains how this document is organized.
Section \ref{sect:configuration}  describes the configuration used for this test.
Section \ref{sect:personnel} describes the necessary roles and lists the individuals assigned to them.
%Section \ref{sect:plannedtestactivities} provides the list of planned test cycles and test cases,
including all relevant information that fully describes the test campaign.

Section \ref{sect:overview} provides a summary of the test results, including an overview in Table \ref{table:summary},
an overall assessment statement and suggestions for possible improvements.
Section \ref{sect:detailedtestresults} provides detailed results for each step in each test case.

The current status of test plan LVV-P69 in Jira is \textbf{ Draft }.

\subsection{References}
\label{sect:references}
\renewcommand{\refname}{}
\bibliography{lsst,refs,books,refs_ads}
\section{Test Configuration}
\label{sect:configuration}

\subsection{Data Collection}

  Observing is not required for this test campaign.

\subsection{Verification Environment}
\label{sect:hwconf}
  An LDF-hosted instance of the Science Platform




\newpage
\section{Personnel}
\label{sect:personnel}

The personnel involved in the test campaign are shown in the following table.

\begin{longtable}{p{3cm}p{3cm}p{3cm}p{6cm}}
\hline
\multicolumn{2}{r}{Test Plan (LVV-P69) owner:} &
\multicolumn{2}{l}{\textbf{ Gregory Dubois-Felsmann } }\\\hline
\multicolumn{2}{r}{ LVV-C151 owner:} &
\multicolumn{2}{l}{\textbf{
    Gregory Dubois-Felsmann
}
} \\\hline
\textbf{Test Case} & \textbf{Assigned to} & \textbf{Executed by} & \textbf{Additional Test Personnel} \\ \hline
\href{https://jira.lsstcorp.org/secure/Tests.jspa#/testCase/LVV-T1818}{LVV-T1818}
& {\small Gregory Dubois-Felsmann } & {\small  } &
\begin{minipage}[]{6cm}
\smallskip
{\small An authorized user of the LSP instance under test }
\medskip
\end{minipage}
\\ \hline
\end{longtable}

\newpage

\section{Test Campaign Overview}
\label{sect:overview}

\subsection{Summary}
\label{sect:summarytable}

\begin{longtable}{p{2cm}p{2.5cm}p{9cm}p{2.5cm}}
\toprule
\multicolumn{3}{p{13.5cm}}{ Test Plan {\bf LVV-P69: DM-SUIT-8 SUIT Portal integrated with workspace }} & Draft \\\hline

  \multicolumn{3}{p{13.5cm}}{ Test Cycle {\bf LVV-C151: DM-SUIT-8 SUIT Portal integrated with workspace }} & Not Executed \\\hline

  {\bf \footnotesize test case} & {\bf \footnotesize status} & {\bf \footnotesize comment} & {\bf \footnotesize issues} \\\toprule

\href{https://jira.lsstcorp.org/secure/Tests.jspa#/testCase/LVV-T1818}{LVV-T1818}
    & Not Executed &
    \begin{minipage}[]{9cm}
    \smallskip
    
    \medskip
    \end{minipage}
    &
    \\\hline
\caption{Test Campaign Summary}
\label{table:summary}
\end{longtable}

\subsection{Overall Assessment}
\label{sect:overallassessment}

Not yet available.

\subsection{Recommended Improvements}
\label{sect:recommendations}

Not yet available.

\newpage
\section{Detailed Test Results}
\label{sect:detailedtestresults}

\subsection{Test Cycle LVV-C151 }

Open test cycle {\it \href{https://jira.lsstcorp.org/secure/Tests.jspa#/testrun/LVV-C151}{DM-SUIT-8 SUIT Portal integrated with workspace}} in Jira.

DM-SUIT-8 SUIT Portal integrated with workspace\\
Status: Not Executed



\subsubsection{Software Version/Baseline}
Not provided.

\subsubsection{Configuration}
Not provided.

\subsubsection{Test Cases in LVV-C151 Test Cycle}

\paragraph{Test Case LVV-T1818 - DM-SUIT-8: Verify Portal integration with workspace (via WebDAV) }\mbox{}\\

Open  \href{https://jira.lsstcorp.org/secure/Tests.jspa#/testCase/LVV-T1818}{\textit{ LVV-T1818 } }
test case in Jira.

This test case verifies that the Portal Aspect software is capable of
accessing a file-oriented workspace via the WebDAV
protocol.\\[2\baselineskip]In so doing, it partially verifies several
Portal Aspect requirements that relate to this capability -
``partially'' because some of these requirements depend on workspace
capabilities which were not present in the prototype WebDAV service
delivered by the DAX group, and also because the milestone was not
envisioned as an exhaustive test covering edge cases:

\begin{itemize}
\tightlist
\item
  DMS-PRTL-REQ-0003 (LVV-9846, Portal access to workspace) is covered at
  ``demonstration'' level, with basic tests of saving image and tabular
  data to the workspace, and only for the User File Workspace ~(there is
  currently no User Database Workspace prototype available);
\item
  DMS-PRTL-REQ-0046 (LVV-9886, Visualization of workspace data) is
  covered at ``demonstration'' level for a couple of FITS image and
  table files, and only for the User File Workspace;
\item
  DMS-PRTL-REQ-0110 (LVV-9954, Tabular data download) is covered at
  ``demonstration'' level, only for catalog data (there was no image
  metadata in the LSP deployment at the time of test), and only for the
  User File Workspace;~
\item
  DMS-PRTL-REQ-0095 (LVV-9932, Saving Displayed Tabular Data) is covered
  at ``demonstration'' level for a simple subset operation in the table
  browser; and
\item
  DMS-PRTL-REQ-0111 (LVV-9951, Image data download) is covered at
  ``demonstration'' level, and only for download from an image display
  screen itself (as LSST-style image metadata services, e.g., ObsTAP,
  were not available in the LSP at the time of testing).
\end{itemize}

\textbf{ Preconditions}:\\
A read/write WebDAV service, accessible to the user performing the test,
must be available and under the same authentication redirect coverage as
the Portal Aspect and the TAP service used for the test.

Execution status: {\bf Not Executed }

Final comment:\\


Detailed steps results:

\begin{longtable}{p{1cm}p{15cm}}
\hline
{Step} & Step Details\\ \hline
1 & Description \\
 & \begin{minipage}[t]{15cm}
{\footnotesize
Using a web browser, navigate to the home page of the selected instance
of the LSP at the LDF. ~From the displayed page, navigate to the Portal
Aspect and log in with valid credentials for the instance under test.

\medskip }
\end{minipage}
\\ \cdashline{2-2}


 & Expected Result \\
 & \begin{minipage}[t]{15cm}{\footnotesize
Successful login to the Portal should display the TAP search screen by
default.

\medskip }
\end{minipage} \\ \cdashline{2-2}

 & Actual Result \\
 & \begin{minipage}[t]{15cm}{\footnotesize

\medskip }
\end{minipage} \\ \cdashline{2-2}

 & Status: \textbf{ Not Executed } \\ \hline

2 & Description \\
 & \begin{minipage}[t]{15cm}
{\footnotesize
Navigate to the TAP search screen, if necessary (in case the default
Portal screen was changed since this test was written), and ensure that
the LSST TAP service associated with the chosen LSP instance is
selected.

\medskip }
\end{minipage}
\\ \cdashline{2-2}


 & Expected Result \\
 & \begin{minipage}[t]{15cm}{\footnotesize
The lists of schemas and tables available in the services should be
displayed as selectable menus.

\medskip }
\end{minipage} \\ \cdashline{2-2}

 & Actual Result \\
 & \begin{minipage}[t]{15cm}{\footnotesize

\medskip }
\end{minipage} \\ \cdashline{2-2}

 & Status: \textbf{ Not Executed } \\ \hline

3 & Description \\
 & \begin{minipage}[t]{15cm}
{\footnotesize
Perform a TAP search on the AllWISE source catalog around the equatorial
coordinates (2, 0) (degrees), with a 30 arcminute radius, using the
Portal UI to specify the query (select the ``Single Table'' radio
button). ~\\
To find the AllWISE source catalog, select the ``wise\_00'' schema from
the schema menu, and then the ``wise\_00.allwise\_p3as\_psd'' table from
the table menu. Use the column selector pane of the search screen to
select the ``ra, decl, source\_id, w1mpro, w2mpro, w3mpro, w4mpro''
columns for retrieval.\\
When the query completes, note the total number of rows in the table, as
displayed in the table header, and record it.

\medskip }
\end{minipage}
\\ \cdashline{2-2}

 & Test Data \\
 & \begin{minipage}[t]{15cm}{\footnotesize
Equatorial coordinates: (2, 0), cone radius 30 arcmin

\medskip }
\end{minipage} \\ \cdashline{2-2}

 & Expected Result \\
 & \begin{minipage}[t]{15cm}{\footnotesize
This query should return about 12,000 rows of data. ~It should be
displayed in a table, as an overlay on a context image, and as a
configurable 2D density plot.

\medskip }
\end{minipage} \\ \cdashline{2-2}

 & Actual Result \\
 & \begin{minipage}[t]{15cm}{\footnotesize

\medskip }
\end{minipage} \\ \cdashline{2-2}

 & Status: \textbf{ Not Executed } \\ \hline

4 & Description \\
 & \begin{minipage}[t]{15cm}
{\footnotesize
Using the table viewer UI, save the result of this search as a text file
in CSV format, specifying that the file be saved to the ``workspace''.
~Ensure that the CSV file's name is ``LVV-T1818-ufw-table.csv''.

\medskip }
\end{minipage}
\\ \cdashline{2-2}


 & Expected Result \\
 & \begin{minipage}[t]{15cm}{\footnotesize
A UI indication that the file has been successfully saved to the
workspace.

\medskip }
\end{minipage} \\ \cdashline{2-2}

 & Actual Result \\
 & \begin{minipage}[t]{15cm}{\footnotesize

\medskip }
\end{minipage} \\ \cdashline{2-2}

 & Status: \textbf{ Not Executed } \\ \hline

5 & Description \\
 & \begin{minipage}[t]{15cm}
{\footnotesize
In the table viewer UI, use the ``funnel'' icon in the table toolbar, if
necessary, to make the column-header filtering text boxes visible.
~Locate the ``w1mpro'' column (band W1 magnitude), enter the filter
expression ``\textless{}8'', and hit TAB or RETURN to apply the filter.
~Note the number of rows remaining following the application of the
filter.

\medskip }
\end{minipage}
\\ \cdashline{2-2}

 & Test Data \\
 & \begin{minipage}[t]{15cm}{\footnotesize
Filter expression ``\textless{} 8'' for the ``w1mpro'' column.

\medskip }
\end{minipage} \\ \cdashline{2-2}

 & Expected Result \\
 & \begin{minipage}[t]{15cm}{\footnotesize
Approximately 10 rows should remain visible.

\medskip }
\end{minipage} \\ \cdashline{2-2}

 & Actual Result \\
 & \begin{minipage}[t]{15cm}{\footnotesize

\medskip }
\end{minipage} \\ \cdashline{2-2}

 & Status: \textbf{ Not Executed } \\ \hline

6 & Description \\
 & \begin{minipage}[t]{15cm}
{\footnotesize
Using the table viewer UI, save the result of this search as a text file
in CSV format, specifying that the file be saved to the ``workspace''.
~Ensure that the CSV file's name is ``LVV-T1818-ufw-table-w1m8.csv''.

\medskip }
\end{minipage}
\\ \cdashline{2-2}


 & Expected Result \\
 & \begin{minipage}[t]{15cm}{\footnotesize
A UI indication that the file has been successfully saved to the
workspace.

\medskip }
\end{minipage} \\ \cdashline{2-2}

 & Actual Result \\
 & \begin{minipage}[t]{15cm}{\footnotesize

\medskip }
\end{minipage} \\ \cdashline{2-2}

 & Status: \textbf{ Not Executed } \\ \hline

7 & Description \\
 & \begin{minipage}[t]{15cm}
{\footnotesize
Navigate to the legacy IRSA image access screen, using the blue
``External Images'' button at the top of the
screen.\\[2\baselineskip]Note that and the following step are being done
exclusively to load an image into the viewer, not itself to demonstrate
an LSP-Portal-specific capability; this test bypasses the lack of a
current image query service in the LSP.

\medskip }
\end{minipage}
\\ \cdashline{2-2}


 & Expected Result \\
 & \begin{minipage}[t]{15cm}{\footnotesize
The ``IRSA Viewer'' image-search screen will be displayed.

\medskip }
\end{minipage} \\ \cdashline{2-2}

 & Actual Result \\
 & \begin{minipage}[t]{15cm}{\footnotesize

\medskip }
\end{minipage} \\ \cdashline{2-2}

 & Status: \textbf{ Not Executed } \\ \hline

8 & Description \\
 & \begin{minipage}[t]{15cm}
{\footnotesize
Use the UI to load a WISE band W1 image for the coordinates (2,0)
without a cutout size limit:

\begin{enumerate}
\tightlist
\item
  Choose Image Type: ``View FITS Images''
\item
  Select Image Source: ``Search''
\item
  Select Target: ``Name or Position'' set to ``2 0''; ``Cutout size''
  left blank
\item
  Select Data Set:

  \begin{enumerate}
  \def\labelenumii{\alph{enumii}.}
  \tightlist
  \item
    If ``WISE AllWISE Atlas'' is not immediately visible under
    ``Selection'', use the ``MISSION'' checkbox on the left to narrow
    the scope to ``WISE''.
  \item
    Use the disclosure triangle to the left of ``WISE AllWISE Atlas'' to
    reveal the filter band selection boxes, and select ``W1''.
  \end{enumerate}
\item
  Click on the ``Search'' button at the bottom of the screen.
\end{enumerate}

\medskip }
\end{minipage}
\\ \cdashline{2-2}

 & Test Data \\
 & \begin{minipage}[t]{15cm}{\footnotesize
Equatorial coordinates (2, 0) expressed as ``2 0'' (``2, 0'' will also
work).

\medskip }
\end{minipage} \\ \cdashline{2-2}

 & Expected Result \\
 & \begin{minipage}[t]{15cm}{\footnotesize
An image for the selected region of sky should be displayed. ~As the
chosen coordinates are not centered in one of the WISE coadded atlas sky
tiles, a ``target'' glyph will be seen displayed off-center at the (2,0)
coordinates.

\medskip }
\end{minipage} \\ \cdashline{2-2}

 & Actual Result \\
 & \begin{minipage}[t]{15cm}{\footnotesize

\medskip }
\end{minipage} \\ \cdashline{2-2}

 & Status: \textbf{ Not Executed } \\ \hline

9 & Description \\
 & \begin{minipage}[t]{15cm}
{\footnotesize
Use the ``save'' action from the image toolbar - the ``floppy disk''
icon at the far left of the toolbar (in the version of Firefly current
at the time of writing) to save the image in FITS format to the
workspace. ~Ensure that the image name is ``LVV-T1818-ufw-image.fits''.

\medskip }
\end{minipage}
\\ \cdashline{2-2}


 & Expected Result \\
 & \begin{minipage}[t]{15cm}{\footnotesize
A UI indication of a successful file-save action.

\medskip }
\end{minipage} \\ \cdashline{2-2}

 & Actual Result \\
 & \begin{minipage}[t]{15cm}{\footnotesize

\medskip }
\end{minipage} \\ \cdashline{2-2}

 & Status: \textbf{ Not Executed } \\ \hline

10 & Description \\
 & \begin{minipage}[t]{15cm}
{\footnotesize
Close the web browser tab or window being used for the test, but do not
quit the browser or clear credentials. ~(That is only for convenience;
it is also acceptable to log out entirely and log in again.)

\medskip }
\end{minipage}
\\ \cdashline{2-2}


 & Expected Result \\
 & \begin{minipage}[t]{15cm}{\footnotesize

\medskip }
\end{minipage} \\ \cdashline{2-2}

 & Actual Result \\
 & \begin{minipage}[t]{15cm}{\footnotesize

\medskip }
\end{minipage} \\ \cdashline{2-2}

 & Status: \textbf{ Not Executed } \\ \hline

11 & Description \\
 & \begin{minipage}[t]{15cm}
{\footnotesize
Using the same web browser, navigate to the home page of the selected
instance of the LSP at the LDF. ~From the displayed page, navigate to
the Portal Aspect.

\medskip }
\end{minipage}
\\ \cdashline{2-2}


 & Expected Result \\
 & \begin{minipage}[t]{15cm}{\footnotesize
No credentials should be needed unless an explicit logout or
credential-clearing action was performed.\\[2\baselineskip]The Portal
Aspect UI should be displayed with the TAP search screen in the
foreground and no image or tabular search results present.

\medskip }
\end{minipage} \\ \cdashline{2-2}

 & Actual Result \\
 & \begin{minipage}[t]{15cm}{\footnotesize

\medskip }
\end{minipage} \\ \cdashline{2-2}

 & Status: \textbf{ Not Executed } \\ \hline

12 & Description \\
 & \begin{minipage}[t]{15cm}
{\footnotesize
Select the blue ``Upload'' button at the top of the screen. ~On the
resulting screen, choose to upload from the workspace. ~Verify that the
image file saved in Step 9 above, ``LVV-T1818-ufw-image.fits'', is
visible. ~Select the file and open it with the UI. ~Verify qualitatively
that it seems to be the same image as displayed above in Step 8.

\medskip }
\end{minipage}
\\ \cdashline{2-2}


 & Expected Result \\
 & \begin{minipage}[t]{15cm}{\footnotesize
The saved image should be displayed. ~Depending on the Portal Aspect
software version deployed at the time of test, the image may be
displayed immediately, or it may be necessary to navigate through a UI
for choosing which extension in the file to display.

\medskip }
\end{minipage} \\ \cdashline{2-2}

 & Actual Result \\
 & \begin{minipage}[t]{15cm}{\footnotesize

\medskip }
\end{minipage} \\ \cdashline{2-2}

 & Status: \textbf{ Not Executed } \\ \hline

13 & Description \\
 & \begin{minipage}[t]{15cm}
{\footnotesize
Select ``Upload'' again. ~This time choose the filtered table file,
``LVV-T1818-ufw-table-w1m8.csv''. ~Note the number of rows displayed and
verify qualitatively that the sky coordinates correspond to the region
around the original (2,0) search center.

\medskip }
\end{minipage}
\\ \cdashline{2-2}


 & Expected Result \\
 & \begin{minipage}[t]{15cm}{\footnotesize
The small, filtered table should be displayed, with the same number of
rows as previously.

\medskip }
\end{minipage} \\ \cdashline{2-2}

 & Actual Result \\
 & \begin{minipage}[t]{15cm}{\footnotesize

\medskip }
\end{minipage} \\ \cdashline{2-2}

 & Status: \textbf{ Not Executed } \\ \hline

14 & Description \\
 & \begin{minipage}[t]{15cm}
{\footnotesize
Select ``Upload'' again. ~This time choose the full table file,
``LVV-T1818-ufw-table.csv''. ~Note the number of rows displayed and
verify qualitatively that the sky coordinates correspond to the region
around the original (2,0) search center.

\medskip }
\end{minipage}
\\ \cdashline{2-2}


 & Expected Result \\
 & \begin{minipage}[t]{15cm}{\footnotesize
The original query table should be displayed, with the same number of
rows as previously.

\medskip }
\end{minipage} \\ \cdashline{2-2}

 & Actual Result \\
 & \begin{minipage}[t]{15cm}{\footnotesize

\medskip }
\end{minipage} \\ \cdashline{2-2}

 & Status: \textbf{ Not Executed } \\ \hline

\end{longtable}


\newpage
\appendix

\input{\lsstDocType-\lsstDocNum.appendix.tex}

\newpage
%Make sure lsst-texmf/bin/generateAcronyms.py is in your path
\section{Acronyms used in this document}\label{sec:acronyms}
\addtocounter{table}{-1}
\begin{longtable}{p{0.145\textwidth}p{0.8\textwidth}}\hline
\textbf{Acronym} & \textbf{Description}  \\\hline

DM & Data Management \\\hline
DMS & Data Management Subsystem \\\hline
LDM & LSST Data Management (Document Handle) \\\hline
LSP & LSST Science Platform \\\hline
LSST & Legacy Survey of Space and Time (formerly Large Synoptic Survey Telescope) \\\hline
PMCS & Project Management Controls System \\\hline
SUIT & Science User Interface and Tools (LSST Data Management WBS element and team, responsible for LSP Portal Aspect) \\\hline
TBD & To Be Defined (Determined) \\\hline
\end{longtable}


\end{document}
